% !TEX root = ../report.tex

\section{Background}\label{background}
\thispagestyle{plain}

This section will explore each of the broad topics in general detail, laying
the basis for the discussion of specific choices in each of the topics in Chapter~\ref{design}.

\subsection{Co-operative robotics}\label{background/coop} % other supermarkets are available

Co-operative robotics has varied definitions across different papers. One such
definition, which generalises co-operative behaviour in robotics, describes it as ``joint
collaborative behaviour that is directed toward some goal in which there is a
common interest or reward''~\cite{barnes1991behaviour}. This description fits the
objectives of this study more appropriately than the specific term swarm robotics, which
has a number of additional requirements, including
that problem solving should be distributed across the swarm~\cite{sahin04}.
This definition does not apply to this project, as the robots designed
here are capable of operating autonomously and will be able to solve certain tasks
individually. In this case, collaboration is used to deliver a performance improvement.
For this reason, the more general term of co-operative robotics will
be used throughout. The aim is to reduce the time taken or increase performance of the system over
a single-robot system~\cite{premvuti1990consideration}. Additionally, by creating a decentralised
and distributed system across a number of homogeneous agents, agent redundancy is introduced which
can improve the completion rate of tasks, especially in potentially volatile
environments~\cite{beckers1994local, parker95}.

Co-operative robotics goes beyond the idea
of collaborative robotics in requiring an additional aspect of intelligence in
the communication and coordination of the individual agents~\cite{cao1995cooperative}.


\subsection{SLAM}\label{background/slam}

An essential skill for mobile robots navigating unknown environments is being
able to build a map of the environment whilst simultaneously identifying its own
position. This is especially true in the absence of external referencing systems
such as GPS.\@ This is known as the Simultaneous Localisation And Mapping (SLAM)
problem and has been one of the most extensively researched topics in mobile
robotics over the last two decades~\cite{grisetti2010tutorial}. As the robot's
estimate of its position is affected by both the previous state's uncertainty
and the any errors in the current measurement, uncertainties in the readings compound
over time. To rectify this, a map with distinctive landmarks can reduce its
localisation error by revisiting these known areas. This is known as loop closure.

SLAM implementations rely on sensor fusion algorithms which take in readings from an
array of sensors and calculates a estimated state changed based on the probability
or error for each sensor. This effectively allows errors between multiple sensors
to be cancelled out, resulting in more reliable estimates. A typical approach is to
use sensor fusion to combine odometry readings from wheel encoders with acceleration
information obtained from an inertial measurement unit (IMU) to correct for errors
caused by wheels slipping and sensor imperfections.

There are a large variety of solutions to suit various system requirements, which
fall into the two categories of filtering and smoothing. Filtering
involves creating a state estimation where the state of the system consists of
the current robot position and the map. The estimate is augmented and improved
by considering the new measurements as they become available. Some popular
approaches to filtering are techniques such as Kalman filters, particle filters
and information filters. This technique is known as online SLAM where only the most
recent pose (the robot's position and orientation) is recovered with previous poses
marginalised out. Smoothing approaches involve a full estimate of the trajectory of
the robot from all available
measurements. These typically use least-square error minimisation techniques and are
used to address the problem known as the full SLAM problem. Full SLAM involves
attempting to map the entire path.

The state of the system is known as $x_k$ and
this can be defined as a function which uses the previous state to determine the next
state. As the model can not be perfectly accurate, there will be uncertainty in the
readings and so this must be taken into account by the model. As a result, $x_k$,
which is known as the motion model, is defined as
\begin{equation}
x_{k} = f(x_{k-1}, q_{k-1})\,,
\end{equation}
where $q_{k-1}$ is the randomness introduced to the system. As such, this can
also be represented by the probability distribution
\begin{equation}
x_{k} \sim p(x_{k} | x_{k-1})\,.
\end{equation}
Both of these imply that the state is stochastic and depends on the previous
state. The probability distribution helps emphasise that the current state is
drawn from a distribution of possible states based on the previous state. Given that a perfect sensor is not possible, the current state will also have noise
in the reading. This is known as the measurement model and can be defined as
\begin{equation}
y_{k} = h(x_{k}, r_{k})\,,
\end{equation}
where $r$ represents the uncertainty that is present in the model of the sensor. As
before this can be expressed as an uncertainty model:
\begin{equation}
y_{k} \sim p(y_{k} | y_{k-1})\,.
\end{equation}
It is assumed that the motion and measurement models are Markovian in that
the current state only depends on the previous state. The measurement model only
depends on the current state and no previous values.

By applying Bayes' theorem and marginalisation the current state can be described as
\begin{align}
\label{eqn:predict}
p(x_{k} | y_{1:k-1}) & = \int p(x_{k}|x_{k-1}) p(x_{k-1} | y_{1:k-1}) dx_{k-1} \\
\label{eqn:update}
p(x_{k} | y_{1:k}) &= \frac{ p(y_{k}|x_{k})p(x_{k}|x_{1:k-1})}{ p(y_{k}|y_{1:k-1})}\,.
\end{align}
Equation~\ref{eqn:predict} is known as the predict equation. By integrating over
the previous state, all potential outcomes of the state $x_k$ are
considered. Equation~\ref{eqn:update} is referred to as the update equation,
as the prediction is updated using the new measurement information~\cite{kam1997sensorfusion}.

One of the most common methods of implementing SLAM is filtering using an
Extended Kalman Filter (EKF). An EKF is an efficient, recursive filter
that estimates the state of a dynamic system from a series of noisy measurements~\cite{fox2003bayesian}.
This uses the premise of the predict and update equations as joint probability
distributions. Given variables defined on a probability space, the joint
probability distribution gives the probability that each of the variables falls in any
particular range or set. It uses these techniques to estimates a state vector containing
both the location of landmarks in the map and the robot pose~\cite{huang2007convergence}.

\subsection{Computer vision}

Computer vision is the analysis of digital image or video data to allow a computer
system to gain a high-level understanding of the 3D environment contained within
the image\cite{CVBallard}.A common application for computer vision is the identification and classification
of objects. Identification generally involves the recognition of features of the
object and the comparison of these features and their relative positions to a
known model of the object. Classification usually involves machine learning
algorithms to build up a definition of the object based on its visible properties\cite{CVpaoletti2018new}.

Computer vision can also be used to triangulate the position of objects in the
field of view by measuring the discrepancy in the objects position in the two camera
frames given a translation matrix relating the two cameras.

Computer vision can be well integrated with SLAM providing a means of both the measure
of distance (if the CV system is bi-ocular) and the identification of distinctive
features in the environment to allow loop closure\cite{CVho2006loop}.
