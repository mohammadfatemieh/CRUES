% !TEX root = ../report.tex

\chapter{Conclusion}\label{conclusion}
This project aimed to investigate co-operative robotics within 
communication restricted environments by constructing several robots 
which used scalable methods to map and explore their surroundings and find 
a goal. The project required the the mechanical construction and 
electrical design of the robot as well as the development of sensor and 
control software.  Research into these topics provided knowledge which 
allowed 
design decisions to be taken for each of the main aspects of the project. 
Agile methodologies were employed to manage the team over the course of 
the project and were proven successful with 5 out of 6 Major objectives 
completed. 

Influenced by the design decisions, a pre-built chassis was 
used as the basis for a mechanically functional robot using odometry, 
exteroceptive sensors and actuators to explore its environment. These sensors included ultrasonic sensors, an IMU, encoders and a basic 
camera each of which were mounted on the robot via a bespoke PCB. The PCB 
design was iterated upon throughout the project to achieve a final design 
which connected each of the components in a mechanically and 
electrically sound manner. 

In order for each of the components to function together, a software 
architecture based on data-oriented software design principles was 
created. This architecture centred on the Robot Operating System (ROS) 
library which uses a ``publish-subscribe'' model to allow modules in the 
system to communicate while minimising coupling. For each of the aforementioned components of the 
system, a software module was written based on ROS node design principles. 
Each of these modules was independently tested extensively. 

To allow the robot to explore, ROS libraries for differential drive, 
sensor-fusion and Simultaneous Localisation and Mapping (SLAM) were used. 
Introducing each of these libraries added complexity to the project and 
issues were encountered throughout integration. Although a number of these 
issues were overcome, the delay in completing these objectives meant that 
the final objective --- the development of an AI control module --- could not 
be completed. 

As a consequence, system testing results obtained using the custom designed testing environment can only be analysed as 
benchmark results which provide a baseline for future work going forward. 
Despite this, overall the project had a myriad of successful outcomes and 
valuable lessons learned.      
