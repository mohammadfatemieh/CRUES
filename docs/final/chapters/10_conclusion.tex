% !TEX root = ../report.tex

\chapter{Conclusion}\label{conclusion}
This project aimed to investigate co-operative robotics within environments
with restricted communication by constructing several robots
which used scalable methods to map and explore their surroundings and find
a goal. The project required the mechanical construction and
electrical design of three robots, as well as the development of sensor and
control software.  Research into these topics provided knowledge which allowed
design decisions to be taken for each of the main aspects of the project.
Aspects of agile methodology were employed to manage the team over the course of
the project and were proven successful with five of the six major objectives
completed. Influenced by the design decisions, a pre-built chassis was
used as the basis for a mechanically functional robot using odometry,
exteroceptive sensors, and actuators to explore its environment. These sensors
included ultrasonic sensors, an IMU, encoders and a basic
camera, each of which were mounted on the robot via a bespoke PCB. The PCB
design was iterated upon throughout the project to achieve a final design
which connected each of the components in a mechanically and
electrically sound manner.

For the components to function together, a software
architecture based on data-oriented software design principles was
created. This architecture centred on the Robot Operating System (ROS)
library which uses the publish--subscribe pattern to allow modules in the
system to communicate while minimising coupling. Software modules based on
ROS node design principles were written for each sensor and actuator, as well
as for control structures. Each of these modules was rigorously tested, both
in isolation and as part of the overall system.

A flexible communication system was developed that allows robots to communicate
directly to a variety of ROS topics, allowing dynamic creation of topics as required.
This will allow communication to occur simultaneously at several control levels,
from global path planning to sensor synchronisation.

To allow the robot to explore, ROS libraries for differential drive kinematics,
sensor fusion, and SLAM were used.
Introducing each of these libraries added complexity to the project, and
issues were encountered throughout integration. Technical challenges were
encountered with mechanical components, power supply, and the synchronisation of
active sensors between robots. Although a number of these
issues were overcome, the delays in completing these objectives meant that
the final objective---the development of an co-operative AI control modules---could
not be completed. As a consequence, system testing results obtained using the
custom-designed testing environment can only be seen as benchmark results which
provide a baseline for future work.

Significant thought has been given to solutions to the outstanding problems, with
a clear path outlined for future development of the project. Substantial progress
was made in constructing a flexible platform for addressing the remaining
challenges. In particular, significant work was accomplished on communication
protocols, sensor fusion, localisation and mapping.
A combination of the configurable testing environment, the robust mechanical
construction of the robot, and the modular software architecture allows for rapid
development of additional control systems.
