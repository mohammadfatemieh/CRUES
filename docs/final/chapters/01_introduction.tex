% !TEX root = ../report.tex

\chapter{Introduction}\label{introduction}
\section{Context}\label{introduction/context}
Using multiple co-operating robots (named and referred to as Blinky, Inky
and Clyde) can be a useful strategy when attempting to complete tasks that
can be divided and parallelised, such as exploration of a large area.
Usually, this requires a centralised control system that can
coordinate the robots to accomplish the task. However, in
environments where radio communication is impossible or severely restricted,
such as underground or in areas with high interference, these systems would
be unable to complete their task. In these circumstances, communication may
be limited to non-telemetric sensors that require line-of-sight. Therefore,
intelligent systems are required so they can co-operate effectively.

Co-operative robotics can be used to solve a variety of tasks more 
effectively than a more complex individual robot, introducing distribution
and redundancy into the system which can be highly advantageous over single 
agent systems~\cite{dudek96}. This redundancy can be mission critical in 
scenarios where a hazardous environment poses risks to individual robots. 
Therefore the loss of any single robot is not fatal to the completion of the task as all
the robots are homogeneous and can complete the task individually.

Previous research into co-operative robotics has focused on UAVs~\cite{khan18},
non-autonomous agents~\cite{jimenez18}, or made use of extensive
communication, such as by using the cloud~\cite{wensing2018cooperative}.
This study will use restricted non-telemetric communication (i.e. local 
communication between neighbouring robots rather than communication over a 
central server). This
application area was inspired in part by a recent incident necessitating cave 
exploration and rescue in Thailand~\cite{bbcthailand}.

To remove the physical complexities of operating on difficult
terrain and allow research to focus on communication and problem-solving 
algorithms, a toy problem has been devised for the robots to solve. This
will take the form of a simple maze which contains a target that needs to
be found and identified by the robots. The primary aim of the study is to
construct multiple robots which can navigate this maze and co-ordinate their
efforts to find the target more quickly than could be achieved by each robot 
individually.

An additional requirement is that the robots should be constructed using
inexpensive components and a Raspberry Pi single board computer (RPi). This is necessitated by the increased cost implied
by the need for multiple robots, in addition to mitigating the risk in the 
event of the robot failure given the potentially dangerous environments.

\section{Objectives}\label{introduction/objectives}
\begin{itemize}
\item{Design a simple differential drive robot capable of exploring and perceiving its environment---Major Objective}
    \item{Construct two robots using this design---Major Objective}
    \item{Implement a Simultaneous Localisation and Mapping (SLAM) algorithm to allow the robots to explore an area---Major Objective}
    \item{Develop a system to allow the robots to interact and communicate with each other---Major Objective}
    \item{Implement algorithms to search a maze which can be dynamically parallelised over any number of agents---Major Objective}
    \item{Develop a test environment and evaluate the robots’ performance---Major Objective}
    \item{Add additional robot(s) and evaluate scalability of approach---Optional Objective}
    \item{Improve SLAM by adding loop closure between robots---Optional Objective}
\end{itemize}
