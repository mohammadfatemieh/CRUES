% !TEX root = ../report.tex

\chapter{Literature Review}\label{litreview}
A literature review was undertaken on all aspects of the project including
co-operative robotics, computer vision and SLAM both in general and more
specifically using basic sensors. This was done initially using the Google
Scholar search engine to search for articles which would be of relevance to
these topics in isolation and also collectively. These articles, which mainly
consisted of a variety of journal articles and conference proceedings from
the IEEE, were then screened further based on their titles and abstracts.
Priority was given to those with direct relation to the study---for example
those using ultrasonic sensors to do SLAM or those also using ground vehicles.
Related articles were also found through citations used within these articles
in addition to the related articles feature provided by Google Scholar. These
searches resulted in full investigations of around 4 articles for each of the main
review topics. When more specific information was required e.g., for specific
algorithms related to the study, the search term was altered to refine the
relevant articles. If at any point throughout the study it was felt that more
reading was required, the selection of articles here were revisited, with
related articles used to fill in any gaps in knowledge from the original
searches. \todo{i dont think we need these paragraphs}

\section{Robotics}\label{litreview/robotics}
A robot is ``a machine capable of carrying out a complex series of actions automatically''~\cite{https://en.oxforddictionaries.com/definition/robot}. Robotics is the ``branch of technology which deals with the design, construction, operation, and application''~\cite{https://www.merriam-webster.com/dictionary/robotics} of robots. Robotics combines a number of fields from mechanical, electrical and software engineering within a single system to achieve its goals. By combining the three major disciplines of artificial intelligence, operations research, and control theory, a resultant intelligent control system is created~\cite{https://ieeexplore.ieee.org/stamp/stamp.jsp?tp=&arnumber=1103278} which can be used for a wide range of robotic applications.   

\subsection{Control}\label{litreview/robotics/control}  
As the robots in this study are homogeneous and should be able to complete tasks individually, control is limited to the control of a single robot in the complete system. Robotic mechanisms form the control system for a robot and connect the fixed parts of the robot together by joints to allow motion between fixed parts~\cite{lynch2017modern}.
Actuation of the joints, usually by motors, imparts forces on the robot which allow it to move and perform a variety of tasks~\cite{lynch2017modern}.
The movement of these ``actuators'' is influenced by a number of sensors which provide the system with information about its environment. These sensors can also influence the movement of the actuators through feedback from previous movement instructions~\cite{lynch2017modern}.    

Sensors can take many forms and provide information about internal and external environmental factors of the robot. External sensors provide information which the robot would otherwise find significantly more difficult to discover, one such example being range sensing. This can take a number of different forms including ultrasound, infra-red, Light Detection And Ranging (LIDAR) and binocular computer vision. Each of these sensors use unique methods to determine the distance from the robot to another object in its environment. Ultrasound sensors use high frequency sound waves which reflect from surfaces and return to the sensor in time t to calculate the distance to the object using the speed of sound. Infra-red sensors work in a similar fashion but with invisible light instead of sound. 

LIDAR is a more advanced use of invisible light to more accurately detect distance using more powerful and precise beams of light than in the infra-red case~\cite{LIDAR}. Binocular vision allows depth perception to take place similar to that allowed by human vision~\cite{read2005early} by perspective-based cues which can only be obtained from at least binocular vision~\cite{
pfautz2002depth}. 

Internal sensors can either be related to the actuators or fixed parts of the robot. Those sensors which are connected to the fixed parts of the robot provide feedback about the entire robot and its local environment. The Inertial Measurement Unit (IMU) provides feedback regarding the acceleration and angular velocity with relation to 6 different axis (x, y, z in both linear and angular planes) or Degrees of Freedom (DOF). This can be combined with information from other sensors in order to reduce the overall error in determining the robot's location with relation to its starting position.  

One such sensor is the encoder which is connected to the wheels and provides only feedback information about this actuator. These sensors are connected directly to the shaft of the motors and provide the control system with the actual distance moved by the motors. This allows adjustments to be made and each of the wheels to be maintained at a constant speed when utilised by the Proportional-Integral-Derivative(PID) controller of the robot.  

\subsection{PID Controller}\label{litreview/robotics/pid}
As mentioned above the PID controller uses feedback from the encoders to maintain a straight line of movement and maintain each of wheels at a constant velocity. The PID controller is a feedback loop which is designed to eliminate errors in the actuator systems by the mathematical description \todo{add the mathematical formula for PID add wheres also}. By tuning $ K_p $, $ K_i $ and $ K_d $ for the system in question, the error in the system can be reduced.  \todo{maybe good intro to when we talk about imlementation of motor controller} 

PID controller are a very common solution to problems where the error of the system (the difference between the current state and the desired state) can be measured. PID stands for Proportional-Integral-Derivative, referring to the function of the error with which the output is calculated. The output of the system is the summation of a term which is proportional to the error, a term which is related to the error integrated over time, and a term which is proportional to the rate of change of the error. This is shown in equation \todo{equation}

$$ O(t) = k_{p}e(t) + k_i\int_{0}^{t}e(t)dt + k_d \frac{\mathrm{d} e(t) }{\mathrm{d} t}$$

If $K_d$ and $K_i$ were both 0, and only the P term was active, the output would change to reduce the error in the system. Imagine a temperature controller, if the temperature was too low, the heater would turn on, and the heater would be turned up higher the greater the difference between the actual temperature and the required temperature. In many systems, this simple implementation is adequate, however it has several limitations. 

Consider for instance the temperature controller example where heat is being lost in the system is equal to the P constant times the error. This would result in a stable state with a potentially large error. To prevent this, the I term can be increased. This will increase the longer an error is present, and preventing stable state errors. 

Again, PI controllers are a common control mechanism, however they often have a problem with overshoot due to system inertia. Again considering the temperature controller, this is where the heater is set to increase the temperature, and the desired temperature is reached. The heater turns off, but is still hotter than the rest of the system, causing the system to get too hot. This is especially damaging in systems where the the control in one direction is passive, e.g. the system can actively heat up, but has to wait for heat loss to cool down. This problem can be mitigated by increasing the D component. This will push the state towards the desired state when the error is increasing (e.g. during over-shoot), and pushes the current state away from the desired state as the error is reducing as to minimise overshoot before the over shoot occurs. They do, however, reduce the response speed of the system~\cite{chen2007linear}.

One method of tuning a PID controller is the Ziegler-Nichols method~\cite{ziegler1942optimum}. This principle originates from prior to autonomous robotic control and has been adapted for this application~\cite{aastrom2004revisiting}. The method experimentally finds the critical gain of $K_p$, $K_u$, from which the frequency of oscillation, $T_u$ can be found. These values can then be used to calculate $K_i$ and $K_d$ which results in a tuned system. However, many different formulae and definitions of the critical gain of $K_p$ exist in literature. This leads to the conclusion that the Ziegler-Nichols method is mathematically imperfect and this will have to be explored experimentally during the project.     

\subsection{Rate Control}\label{litreview/robotics/ratecontrol}
Opposed to traditional PID, Rate Control employs only the P and D portions of the equation and uses the current velocity as the base rate as opposed to 0~\cite{koditschek1987quadratic}. This allows the controller to maintain the speed more smoothly and fewer spikes in motor current are required for the required drive~\cite{kawamura1988local}. By removing the $I$ term, the speed will climb more slowly from start-up and recover from dips in speed more slowly. This is mitigated, however, by using the current speed as the base and therefore, a high gain is not required to increase the speed to the required level. 


\section{Cooperative Robotics}\label{litreview/coop}
Co-operative robotics has varied definitions across different papers. One such
definition, which generalises co-operative behaviour in robotics, describes it as ``joint
collaborative behaviour that is directed toward some goal in which there is a
common interest or reward''~\cite{barnes1991behaviour}. This description fits the
objectives of this study more appropriately than the specific term swarm robotics, which
has a number of additional requirements, including
that problem solving should be distributed across the swarm~\cite{sahin04}.
This definition does not apply to this project, as the robots designed
here are capable of operating autonomously and will be able to solve certain tasks
individually. In this case, collaboration is used to deliver a performance improvement.
For this reason, the more general term of co-operative robotics will
be used throughout. The aim is to reduce the time taken or increase performance of the system over
a single-robot system~\cite{premvuti1990consideration}. Additionally, by creating a decentralised
and distributed system across a number of homogeneous agents, agent redundancy is introduced which
can improve the completion rate of tasks, especially in potentially volatile
environments~\cite{beckers1994local, parker95}.

Co-operative robotics goes beyond the idea
of collaborative robotics in requiring an additional aspect of intelligence in
the communication and coordination of the individual agents~\cite{cao1995cooperative}.

\section{SLAM}\label{litreview/slam}
A key challenge in mobile robotics is for the robot to know its own position in the
environment whilst still being able to build a map of its surroundings. 
This is especially true in the absence of external referencing systems
such as GPS to aid in knowing its relative position. This is known as the Simultaneous Localisation And Mapping (SLAM)
problem and has been one of the most extensively researched topics in mobile
robotics over the last two decades~\cite{grisetti2010tutorial}. As the robot's
estimate of its position is affected by both the previous state's uncertainty
and any errors in the current measurement, the uncertainties compound
over time. To rectify this, a map with distinctive landmarks can reduce its
localisation error by revisiting these known areas. This is known as loop closure.

SLAM implementations rely on sensor fusion algorithms as part of their implementation. 
This takes in readings from an array of sensors and calculates an estimated state change
based on the probability of error for each sensor. This effectively allows errors between 
multiple sensors to be cancelled out, resulting in more reliable estimates. A standard 
approach is to use sensor fusion to combine odometry readings from wheel encoders with 
acceleration information obtained from an inertial measurement unit (IMU) to correct for 
errors caused by wheels slipping and sensor imperfections.

There are a large variety of solutions to suit various system requirements, these
can be categorized as either filtering and smoothing. Filtering creates a state estimation 
using the current robot position and the map. The estimate is augmented and improved
by using the new measurements as they become available. Some popular
approaches to filtering are techniques such as Kalman filters, particle filters
and information filters. Smoothing techniques involve a full estimate of the trajectory of
the robot from all available measurements. These typically use least-square error 
minimisation techniques and are used to address the problem known as the full SLAM 
problem which attempts to map the entire path.

The state of the system is known as $x_k$ which is a function that uses the previous 
state to determine the next state. As this can not be perfectly accurate, there will 
be uncertainty in the readings that must be taken into account. As a result, $x_k$,
which is known as the motion model, is defined as
\begin{equation}
x_{k} = f(x_{k-1}, q_{k-1})\,,
\end{equation}
where $q_{k-1}$ is the randomness introduced to the system. As such, this can
also be represented by the probability distribution
\begin{equation}
x_{k} \sim p(x_{k} | x_{k-1})\,.
\end{equation}
Both of these imply that the state is stochastic and depends on the previous
state. The probability distribution emphasises that the current state is
drawn from a distribution of possible states based on the previous state. Given that a perfect sensor is not possible, the current state will also have noise
in the reading. This is known as the measurement model and can be defined as
\begin{equation}
y_{k} = h(x_{k}, r_{k})\,,
\end{equation}
where $r$ represents the uncertainty of the sensor. As
before this can be expressed as an uncertainty model:
\begin{equation}
y_{k} \sim p(y_{k} | y_{k-1})\,.
\end{equation}
It is assumed that the motion and measurement models are Markovian in that
the current state only depends on the previous state. The measurement model only
depends on the current state and no previous values.

By applying Bayes' theorem and marginalisation the current state can be described as
\begin{align}
\label{eqn:predict}
p(x_{k} | y_{1:k-1}) & = \int p(x_{k}|x_{k-1}) p(x_{k-1} | y_{1:k-1}) dx_{k-1} \\
\label{eqn:update}
p(x_{k} | y_{1:k}) &= \frac{ p(y_{k}|x_{k})p(x_{k}|x_{1:k-1})}{ p(y_{k}|y_{1:k-1})}\,.
\end{align}
Equation~\ref{eqn:predict} is known as the predict equation. By integrating over
the previous state, all potential outcomes of the state $x_k$ are
considered. Equation~\ref{eqn:update} is referred to as the update equation,
as the prediction is updated using the new measurement information~\cite{kam1997sensorfusion}.

One of the most common methods of implementing SLAM is filtering using an
Extended Kalman Filter (EKF). An EKF is an efficient, recursive filter
that estimates the state of a dynamic system from a series of noisy measurements~\cite{fox2003bayesian}.
This uses the premise of the predict and update equations as joint probability
distributions. Given variables defined on a probability space, the joint
probability distribution gives the probability that each of the variables falls in any
particular range or set. It uses these techniques to estimates a state vector containing
both the location of landmarks in the map and the robot pose~\cite{huang2007convergence}.


\section{Computer Vision}\label{litreview/cv}
Computer vision is the analysis of digital image or video data to allow a computer
system to gain a high-level understanding of the 3D environment contained within
the image\cite{CVBallard}.A common application for computer vision is the identification and classification
of objects. Identification generally involves the recognition of features of the
object and the comparison of these features and their relative positions to a
known model of the object. Classification usually involves machine learning
algorithms to build up a definition of the object based on its visible properties\cite{CVpaoletti2018new}.

Computer vision can also be used to triangulate the position of objects in the
field of view by measuring the discrepancy in the objects position in the two camera
frames given a translation matrix relating the two cameras.

Computer vision can be well integrated with SLAM providing a means of both the measure
of distance (if the CV system is bi-ocular) and the identification of distinctive
features in the environment to allow loop closure\cite{CVho2006loop}.

\section{ROS}\label{litreview/ROS}
The Robot Operating System (ROS) is a framework for developing robot
software designed to allow flexible software design. It is a collection of tools, 
libraries and conventions that aim to simplify creating complex and 
robust robotic systems across a variety of platforms.\cite{aboutROS}
ROS was designed with the objective of being as modular and distributed
as possible. This modularity allows the user to be able to use as much or
as little of ROS as they desire, where their own implementation can be
easily fit into the system. \cite{rosForMe}

ROS uses a peer-to-peer networking topology to communicate throughout 
the system. These systems consist of a number of processes called nodes 
that preform the system's computation where these nodes are likely 
running across multiple machines. The peer-to-peer  topology requires 
some sort of lookup mechanism to allow processes to find each other at 
runtime. Nodes communicate by passing messages that are data structures 
of typed fields. These messages can consist of arbitrarily nested 
structures and arrays. \cite{crick2017rosbridge}

Nodes can use two types of communication to send messages within the ROS 
framework, services and topics. Services are synchronous and are like function
calls in traditional programming languages, where only one node in the 
system can provide a service of a specific name. Alternatively, topics are
asynchronous streams of objects published by a node. Other nodes can 
subscribe by creating  handler function when a new data object is available.
Multiple nodes can concurrently publish and/or subscribe to the same topic and
a single node may publish and/or subscribe to multiple topics.

ROS was also designed to be language-neutral and supports languages 
such as C++, Python and Octave. To support this cross-language 
development, ROS uses a simple, language-neutral interface definition 
language (IDL) to describe the messages sent between modules 
\cite{quigley2009ros}. Code generators for each language then generate 
native implementations which are serialised and de-serialised by ROS 
as messages are sent and received. This results in a language-neutral 
message processing scheme where languages can be used as the programmer 
prefers based on the requirements of that given module. 

There are other alternatives to ROS, such as Yet Another Robot Platform 
(YARP). YARP was designed to attempt to make robot software more stable 
and long-lasting without compromising flexibility to change the sensors, 
processors and actuators. It also communicates using a peer-to-peer 
topology with an extensible family of connection types however, YARP is 
written in C++ and does not support other languages \cite{aboutYARP}.
The YARP model of communication is transport-neutral, meaning the details 
of the underlying networks and protocols in use are decoupled from the 
data flow \cite{exactlyIsYARP}. Compared to ROS, YARP is less widely used.
As a result it does not have the same extensive a range of libraries available 
which implement commonly required functionality for a robotic system.

\section{AI}\label{litreview/maze}
Artificial intelligence is another component of intelligent robotic control and is the development of computer systems able to perform tasks normally requiring human intelligence~\cite{russell2016artificial}. Searching problems, and the algorithms which solve them, are a common branch of artificial intelligence into which much research has been undertaken. Search algorithms and optimisation algorithms can be thought of as highly similar, and in the case of a weighted tree search either can be used to solve the problem~\cite{kanal2012search}. 
\subsection{Maze Exploration}\label{litreview/maze/exploration}
In robotics, the exploration of an unknown environment to gain knowledge, 
is a well researched problem. Robot exploration is particularly important for 
environments that are difficult or dangerous for humans. There are many
algorithms that can be used for exploration such as Wall Follower, 
Trémaux's and Pledge.

One of the simplest exploration algorithms is the Wall Follower algorithm, 
also known as the right-hand rule, when prioritising turning right or 
left-hand rule when prioritising turning left. If the maze is 
simply connected, that is, any two points of the maze can be traversed without
leaving the maze, then the Wall Follower algorithm is guaranteed to reach a
goal/exit if there is one. Otherwise, the algorithm will return to the starting 
point having traversed every corridor at least once \cite{wallFollowerArcBotics}.
The steps of the algorithm are relatively straightforward. If using the
right-hand rule:
\begin{enumerate}
\item If the robot can turn right, turn right
\item \textbf{Else} continue going straight on
\item \textbf{Else} turn left if possible
\item \textbf{Else} a dead end is reached, so turn $180^{\circ}$
\end{enumerate}
These steps ensure to keep a wall on the right hand side of the robot at 
all times. The left-rule is just the opposite where the robot turns left 
if at all possible to keep the wall on the robots' left. As a result, 
this algorithm is inherently inefficiently as it exhaustively searches the
maze. In many cases, one of these algorithms would be significantly quicker
than the other but that is impossible to know without prior knowledge of 
the maze.

An alternative to the Wall Follower algorithm is Pledge's algorithm. This
aims to solve the problem where Wall Follower could be stuck in a loop. An 
example of which is if walls form to create a rectangle in the centre of 
the maze, the robot would continually turn right or left (depending on
algorithm) around that rectangle, never to find the exit. Pledge solves this
by keeping a count of the number of turns made (if not right-angled turns,
then the angle of turn is used instead) as well as the initial direction of
travel. \cite{klein2011pledge}

\begin{algorithm}
\caption{Pledge's Algorithm}
\begin{algorithmic}
\STATE Set angle counter to 0
\REPEAT
\REPEAT
\STATE Walk straight ahead;
\UNTIL {wall hit};
\STATE Turn right;
\REPEAT
\STATE Follow the obstacle's wall;
\UNTIL{angle counter = 0;}
\UNTIL {exit found;}
\end{algorithmic}
\end{algorithm}
Using the initial direction and counting the turns made, allows the 
algorithm to avoid traps such as an upper case letter "G" that simpler 
algorithms struggle with.

The Trémaux maze-solving algorithm requires the robot to record its path 
throughout its navigation routine for finding a route. Any given path 
can be unmarked, marked once or marked twice. If a path is marked twice that 
indicates the robot has travelled down it in both directions. The robot 
will not travel down any path marked twice for a third time, therefore 
treating them as dead-ends. Paths without markings are prioritised, before 
choosing those marked once in search of further unmarked paths. If a 
solution is found then the path that is marked only once is the route 
from the goal back to the start point. Otherwise if no solution is found,
all paths in the maze will be marked twice \cite{even2011graph}.
Importantly, this will likely not find the shortest path but it will be 
guaranteed to find one if it exists.

\subsection{Path Finding}\label{litreview/maze/path}
There are many algorithms that can be used to find the shortest path 
through a graph such as Dijkstra's algorithm and A* search. In the 
context of our agents, path finding is likely to be used once 
the goal has been found to return the robot back to its starting 
position. In the case of an unweighted graph, one of the simplest algorithm 
is breadth-first search. This explores all neighbours at the same depth 
before moving onto nodes at the next depth level. This would be guaranteed 
to find a path with the shortest number of edges and is $\mathcal{O}(|V| + |E|)$ 
where $|V|$ is the number of vertices and $|E|$ is the number of edges. 
Breadth-first search searches the oldest discovered node first meaning 
a large number of backtracks to reach the oldest discovered node. As 
backtracking in physical space is expensive, this results in a very 
costly search for a single-agent. As more agents are added, the cost of this 
decreases as less backtracking would take place. 

In the case where the edges have a weighting, Dijkstra's algorithm is one 
of the most commonly used methods for finding the shortest path between a 
specified pair of nodes in the graph. The steps for Dijkstra's algorithm are:
\begin{enumerate}
\item Assign the start node to have a distance of 0 and all other nodes to 
have a distance of $\infty$
\item Search the unvisited nodes for the closest node. Stop if the closest 
node is $\infty$ (means there is no path to target node). This node is now 
labelled as visited and as the current node.
\item For each node adjacent to the new current that is unvisited, update 
the saved distance to be the sum of the edge weight and the distance from 
start to the current node if this is less than the saved total distance to 
that node. 
\item Repeat the process from Step 2, doing as many iterations as required.
\end{enumerate}
This version of Dijkstra's algorithm runs in $\mathcal{O}(|V|^{2})$ 
\cite{xu2007improved}. An improvement to this original algorithm using a 
min-priority queue implemented by a Fibonacci heap, runs in $\mathcal{O}
(|E| + |V|log |V|)$ and was first proposed in \cite{fredman1987fibonacci}.

Developed initially as an extension of Dijkstra's algorithm, A* is a 
pathfinding algorithm that typically achieves better performance by 
use of a heuristic. At each node, A* chooses the node which it believes 
can be used in the shortest path. It does so by choosing the node with 
the lowest value of f, where f is the sum of g and h, where g is the 
distance to the current node plus the known distance to that adjacent 
node and h is the heuristic value from that node to the goal. The heuristic 
value can be found exactly as the first step of the algorithm though 
this is very time consuming so approximate heuristics are used such 
as the Manhattan distance, Diagonal distance and the Euclidean distance. 