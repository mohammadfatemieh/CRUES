% !TEX root = ../report.tex

\chapter{Background}\label{litreview}

\section{Co-operative Robotics}\label{litreview/robotics}
Robotics is the branch of technology which deals with the design, construction, 
operation, and application of machines capable of carrying out a complex series of 
actions automatically~\cite{robotdef, roboticsdef}. Robotics 
combines a number of fields from mechanical,
electrical and software engineering within a single system to achieve its goals. By 
combining the three major disciplines of artificial intelligence, operations 
research, and control theory, a resultant intelligent control system is created~\cite{saridis1983intelligent} which can be 
used for a wide range of robotic applications.

Co-operative robotics has varied definitions across different papers. One such
definition, which generalises co-operative behaviour in robotics, describes it 
as ``joint collaborative behaviour that is directed toward some goal in which 
there is a common interest or reward''~\cite{barnes1991behaviour}. This 
description fits the objectives of this study more appropriately than the 
specific term ``swarm robotics'',  which has a number of additional requirements, 
including that problem solving should be distributed across the swarm~\cite{sahin04}. This definition does not apply to this project, as the robots 
described herein are capable of operating autonomously and will be able to solve 
certain tasks individually. In this case, collaboration is used to deliver a 
performance improvement. For this reason, the more general term of co-operative 
robotics will be used throughout. 

The aim of co-operative robotics is to  develop multi-robot systems which have improved performance over single-robot systems~\cite{premvuti1990consideration}. Additionally, by creating a decentralised and 
distributed system across several homogeneous agents, agent redundancy is 
introduced which can improve the completion rate of tasks, especially in 
potentially volatile environments~\cite{beckers1994local, parker95}.
Co-operative robotics goes beyond the idea of collaborative robotics in
requiring an additional aspect of intelligence in the communication and
coordination of the individual agents~\cite{cao1995cooperative}.

Co-operative robotics is an area of active research, and is growing in popularity---as the
capabilities of the technology increases---as it can be used in a wide-variety of
situations. In 2011, co-operative robotics was used as part of the
response to the Great Eastern Japan Earthquake. A team from Kyoto University
used co-operating, remotely operated, underwater vehicles to search for submerged
debris to assist with resuming fishing in the region~\cite{matsuno2014utilization}.
Although co-operative, their systems still required a level of human control.
Dr Nithin Mathews' research in 2012 provided a communication system without
human interaction~\cite{mathews2012spatially}. This allowed many smaller robots
to coordinate to complete tasks that one could not on its own, such as push a
larger object. Their system, however, requires a global view of the area to
coordinate their efforts. However, since these findings, the progress
in the field has accelerated. An excellent example is Sebastien de Rivas' work at
Harvard University~\cite{rollsroyceSWARM}. In partnership with Rolls Royce, his
team have spent the last eight years developing very lightweight co-operating robots.
Currently, weighing \SI{1.5}{\g} and measuring \SI{4.5}{\cm} in length, the
four-legged micro-robots have eight degrees of freedom, and are being developed with
the aim of reducing their size so they can inspect the internal workings of an engine.

\section{Robotic Control} \label{litreview/robotics/control}  
As the robots in this study are homogeneous and should be able to complete 
tasks individually, control is limited to the control of a single robot in the 
complete system. Robotic mechanisms form the control system for a robot and 
connect the fixed parts of the robot together by joints, allowing motion between 
these fixed parts~\cite{lynch2017modern}. Actuation of the joints, usually 
by motors, imparts forces on the robot which allow it to move and perform a 
variety of tasks~\cite{lynch2017modern}. The movement of these actuators 
is influenced by several sensors which provide the system with information 
about its environment. These sensors can also influence the movement of the 
actuators through feedback from previous movement instructions~\cite{lynch2017modern}.    

Robots generally have both exteroceptive and proprioceptive sensors. Exteroceptive sensors provide information describing the robot's environment, such as a range sensor. These can take a number of different forms 
including ultrasound, infrared, Light Detection And Ranging (LIDAR) and 
stereo computer vision. Each of these sensors use unique methods to determine 
the distance from the robot to another object in its environment. Ultrasound 
sensors use high frequency sound waves which reflect from surfaces and return 
to the sensor. The time taken to return and the speed of sound is used to 
calculate the distance to the object. Infrared sensors work in a similar 
fashion but with invisible light instead of sound. 

LIDAR is a more advanced use of invisible light to more accurately detect 
distance using more powerful and precise beams of light than in the infrared 
case~\cite{lidar}. Binocular vision allows depth perception to take place 
similar to that allowed by human vision by perspective-based cues~\cite{read2005early, pfautz2002depth}. 

Proprioceptive sensors can either be related to the actuators or fixed parts of the 
robot. One such sensor is the encoder which is connected to the wheels and provides 
only feedback information about this actuator. These sensors are connected 
directly to the shaft of the motors and provide the control system with the 
actual distance moved by the motors. This allows adjustments to be made and each 
of the wheels to be maintained at a constant speed when utilised by the 
Proportional-Integral-Derivative(PID) controller of a robot. 


The Inertial Measurement Unit (IMU) is another example. It provides feedback regarding the acceleration and angular 
velocity with relation to 6 different axes (x, y, z in both linear and angular 
planes) or Degrees of Freedom (DOF). This can be combined with information from 
other sensors in order to reduce the overall error in determining the robot's 
location relative to its starting position.  

\subsection{PID Controller}\label{litreview/robotics/pid}
As mentioned above, PID controllers are a very common solution to problems where the error of the
system (the difference between the current state and the desired state) can be
measured. Proportional-Integral-Derivative refers to the
function of the error with which the output is calculated. The output of the
system is the summation of a term which is proportional to the error, a term
which is related to the error integrated over time, and a term which is
proportional to the rate of change of the error~\cite{aastrom2006advanced}. This is shown by Equation \ref{eqn:pid}.

\begin{equation}
\label{eqn:pid}
O(t) = k_{p}e(t) + k_i\int_{0}^{t}e(t)dt + k_d \frac{\mathrm{d} e(t) }{\mathrm{d} t}
\end{equation}

If $K_d$ and $K_i$ were both 0, and only the P term was active, the output
would change to reduce the error in the system. For example, with a
temperature controller, if the temperature was too low the heater would
turn on, with the difference between the actual temperature and the required
temperature used to adjust how high the heater was set. In many systems, this simple
implementation is adequate, however it has several limitations.

Consider for instance the temperature controller example where heat is being
lost in the system is equal to $K_{p}e(t)$. This would
result in a stable state with a potentially large error. To prevent this,
the I term can be increased. This will increase the longer an error is
present, thus preventing stable state errors.

Again, PI controllers are a common control mechanism, however they often have
a problem with overshoot due to system inertia. Once more considering the
temperature controller, this is where the heater is set to increase the
temperature, and the desired temperature is reached. The heater turns off, but
is still hotter than the rest of the system, causing the system to get too hot.
This is especially damaging in systems where the control in one direction
is passive, e.g. the system can actively heat up, but must wait for heat loss
to cool down. This problem can be mitigated by increasing the D component.
This will push the state towards the desired state when the error is increasing
(e.g. during over-shoot), and pushes the current state away from the desired
state as the error is reducing as to minimise overshoot before it
occurs. This does, however, reduce the response speed of the system~\cite{chen2007linear}.

One method of tuning a PID controller is the Ziegler-Nichols method~\cite{ziegler1942optimum}. This principle originates from before autonomous
robotic control and has been adapted for this application~\cite{aastrom2004revisiting}. The method experimentally finds the critical gain, $K_u$, which is a value of $K_p$
where steady oscillation occurs. The frequency of oscillation, $T_u$, is then found and
these values can then be used to calculate $K_i$ and $K_d$ which results in a
tuned system. However, many different formulae and definitions of the critical
gain of $K_p$ exist in literature. This leads to the conclusion that significant experimental effort will still need to be invested in the process of tuning the PID.

\section{SLAM}\label{litreview/slam}
A key challenge in mobile robotics is for the robot to know its own position in the
environment whilst still being able to build a map of its surroundings.
This is especially true in the absence of external referencing systems
such as GPS to aid in knowing its relative position. This is known as the Simultaneous Localisation And Mapping (SLAM)
problem and has been one of the most extensively researched topics in mobile
robotics over the last two decades~\cite{grisetti2010tutorial}. As the robot's
estimate of its position is affected by both the previous state's uncertainty
and any errors in the current measurement, the uncertainties compound
over time. To rectify this, a map with distinctive landmarks can reduce its
localisation error by revisiting these known areas. This is known as loop closure.

SLAM generally rely on sensor fusion algorithms as part of their implementation.
These take in readings from an array of sensors and calculate an estimated state change
based on the probability of error for each sensor. This mitigates errors, resulting in
more reliable estimates than is possible with either one sensor. A standard
approach is to use sensor fusion to combine odometry readings from wheel encoders with
acceleration information obtained from an inertial measurement unit (IMU) to correct for
errors caused by wheels slipping and sensor imperfections.

There are a large variety of solutions to suit various system requirements, these
can be categorized as either filtering and smoothing. Filtering creates a state estimation
using the current robot position and the map. The estimate is augmented and improved
by using the new measurements as they become available. Some popular
approaches to filtering are techniques such as Kalman filters, particle filters
and information filters. Smoothing techniques involve a full estimate of the trajectory of
the robot from all available measurements. These typically use least-square error
minimisation techniques and are used to address the problem known as the full SLAM
problem which attempts to map the entire path.

The state of the system is known as $x_k$ which is a function that uses the previous
state to determine the next state. As this can not be perfectly accurate, there will
be uncertainty in the readings that must be considered. As a result, $x_k$,
which is known as the motion model, is defined as
\begin{equation}
x_{k} = f(x_{k-1}, q_{k-1})\,,
\end{equation}
where $q_{k-1}$ is the randomness introduced to the system. As such, this can
also be represented by the probability distribution
\begin{equation}
x_{k} \sim p(x_{k} | x_{k-1})\,.
\end{equation}
Both equations imply that the state is stochastic and depends on the previous
state. The probability distribution emphasises that the current state is
drawn from a distribution of possible states based on the previous state. Given that a perfect sensor is not possible, the current state will also have noise
in the reading. This is known as the measurement model and can be defined as
\begin{equation}
y_{k} = h(x_{k}, r_{k})\,,
\end{equation}
where $r$ represents the uncertainty of the sensor. As
before this can be expressed as an uncertainty model:
\begin{equation}
y_{k} \sim p(y_{k} | y_{k-1})\,.
\end{equation}
It is assumed that the motion and measurement models are Markovian in that
the current state only depends on the previous state. The measurement model only
depends on the current state and no previous values.

By applying Bayes' theorem and marginalisation the current state can be described as
\begin{align}
\label{eqn:predict}
p(x_{k} | y_{1:k-1}) & = \int p(x_{k}|x_{k-1}) p(x_{k-1} | y_{1:k-1}) dx_{k-1} \\
\label{eqn:update}
p(x_{k} | y_{1:k}) &= \frac{ p(y_{k}|x_{k})p(x_{k}|x_{1:k-1})}{ p(y_{k}|y_{1:k-1})}\,.
\end{align}
Equation~\ref{eqn:predict} is known as the predict equation. By integrating over
the previous state, all potential outcomes of the state $x_k$ are
considered. Equation~\ref{eqn:update} is referred to as the update equation,
as the prediction is updated using the new measurement information~\cite{kam1997sensorfusion}.

One of the most common methods of implementing SLAM is filtering using an
Extended Kalman Filter (EKF). An EKF is an efficient, recursive filter
that estimates the state of a dynamic system from a series of noisy measurements~\cite{fox2003bayesian}.
This uses the premise of the predict and update equations as joint probability
distributions. Given variables defined on a probability space, the joint
probability distribution gives the probability that each of the variables falls in any
range or set. It uses these techniques to estimate a state vector containing
both the location of landmarks in the map and the robot pose~\cite{huang2007convergence}.


\section{Computer Vision}\label{litreview/cv}
Computer vision is the analysis of digital image or video frames to allow a computer
system to gain a high-level understanding of the 3D environment contained within
the image~\cite{CVBallard}. A common application for computer vision is the identification and classification
of objects. Identification generally involves the recognition of features of the
object and the comparison of these features and their relative positions to a
known model of the object. Classification usually involves machine learning
algorithms to build up a definition of the object based on its visible properties~\cite{CVpaoletti2018new}.

Computer vision can also be used to triangulate the position of objects in the
field of view by measuring the discrepancy in the object's position in the two camera
frames given a translation matrix relating the two cameras.

Computer vision can be well integrated with SLAM providing a means of both the measure
of distance and the identification of distinctive
features in the environment to allow loop closure\cite{CVho2006loop}.

\subsection{Object Detection}\label{litreview/cv/objDet}
\subsubsection{CNN}\label{litreview/cv/objDet/CNN}
A common method for object detection is to use a Convolutional
Neural Network (CNN)~\cite{schmidhuber2015deep}. This is a deep learning method that
considers the relative position of pixels in the image. It
works by scanning a Locally Receptive Field (LRF) across the
image, and searching this small group of pixels for patterns
at each position as it moves. This results in a set of
matrices of outputs, which can then be scanned again. With
each iteration, the complexity of pattern which can be detected
increases. The output of these convolutional layers is then
pooled (combined to reduce size), and used as the input to a
neural network which can then classify the image.


\subsubsection{Feature Based}\label{litreview/cv/objDet/fb}
Another approach to object detection is feature based. This
involves the identification and comparison of key points in an
image of the target object and in the camera frame~\cite{lowe2004distinctive}. There are
several key point detection algorithms---SIFT, BRISK and ORB,
to name a few---each of which has its own way of describing
patterns in the pixels and selecting which will be rarer and
therefore, more useful when identifying objects. Each also has
a method of quantifying the similarity of two matches called a
distance metric. The algorithm then looks for features in both
the frame and the object image which score low in the distance
metric (indicating similarity), filters the matches to remove
false positives, and then uses the relative positions of the
matches to estimate an outline of the object in the image
frame.

\section{ROS}\label{litreview/ROS}
The Robot Operating System (ROS) is a framework for developing robot
software designed to allow flexible software design. It is a collection of tools,
libraries and conventions that aim to simplify creating complex and robust robotic systems across a variety of platforms~\cite{aboutROS}.
It was designed with the objective of being as modular and distributed
as possible. This modularity allows the user to be able to use as much or
as little of ROS as they desire, where their own implementation can be
easily fit into the system~\cite{rosForMe}.

ROS uses a peer-to-peer networking topology to allow communication
throughout the system. These systems consist of a number of processes
that perform the system's computation called nodes which can
run across multiple machines. The peer-to-peer topology requires
a lookup mechanism to allow processes to find other process' addresses at
runtime so they can communicate. Nodes communicate by passing messages
that are data structures of typed fields which can include arbitrarily nested
structures and arrays~\cite{crick2017rosbridge}.


Nodes can use two distinct ROS frameworks, services and topics.
Services are synchronous and perform like function
calls in traditional programming languages, where only one node in the
system can provide a service of a specific name. Alternatively, topics are
asynchronous streams of objects published by a node. Other nodes can
subscribe by creating a handler function when a new data object is available.
Multiple nodes can concurrently publish and/or subscribe to the same topic and
a single node may publish and/or subscribe to multiple topics.

ROS was also designed to be language-neutral and supports languages
such as C++, Python and Octave. To support this cross-language
development, ROS uses a simple, language-neutral Interface Definition
Language (IDL) to describe the messages sent between modules
\cite{quigley2009ros}. Code generators for each language then generate
native implementations which are serialised and de-serialised by ROS
as messages are sent and received. This results in a language-neutral
message processing scheme where languages can be used as the programmer
prefers based on the requirements of that given module.

There are other alternatives to ROS, such as Yet Another Robot Platform
(YARP). YARP was designed to attempt to make robot software more stable
and long-lasting without compromising flexibility to change the sensors,
processors and actuators. It also communicates using a peer-to-peer
topology with an extensible family of connection types. However, YARP is
written in C++ and does not support other languages \cite{aboutYARP}.
The YARP model of communication is transport-neutral, meaning the details
of the underlying networks and protocols in use are decoupled from the
data flow \cite{exactlyIsYARP}. Compared to ROS, YARP is less widely used.
As a result, it does not have the same extensive range of libraries available
which implement commonly required functionality for a robotic system.

\section{AI}\label{litreview/maze}
Artificial intelligence is another component of intelligent robotic control and is the
development of computer systems able to perform tasks normally requiring human intelligence~\cite{russell2016artificial}. Searching problems, and the algorithms which solve them, are a
common branch of artificial intelligence, into which much research has been undertaken. Search
algorithms and optimisation algorithms can be thought of as highly similar, and in the case of
a weighted tree search, which maze exploration can be modelled as, either can be used to solve the problem~\cite{kanal2012search}.

\subsection{Maze Exploration}\label{litreview/maze/exploration}
The exploration of an unknown environment is a well-researched problem in
robotics. Robot exploration is particularly important for
environments that are difficult or dangerous for humans. There are many
algorithms that can be used for exploration such as Wall Follower,
Trémaux's and Pledge.

One of the simplest exploration algorithms is the Wall Follower algorithm,
also known as the right-hand rule, when prioritising turning right, or
left-hand rule, when prioritising turning left. If the maze is simply
connected, that is there are no loops in the maze, then the Wall Follower
algorithm is guaranteed to reach a goal/exit if one exists. Otherwise, the
algorithm will return to the starting point having traversed every corridor at
least once~\cite{wallFollowerArcBotics}. The steps of the algorithm are
relatively straightforward. The right-hand rule algorithm is shown in Algorithm~\ref{alg:wf}.

\begin{algorithm}
\caption{Wall Follower Algorithm}\label{alg:wf}
\begin{algorithmic}
\REPEAT
\IF{robot can turn right}
  \STATE turn right
\ELSIF{robot can go straight on}
  \STATE go straight on
\ELSIF{robot can turn left}
  \STATE turn left
\ELSE
  \STATE dead end reached, turn
\ENDIF
\UNTIL{goal is reached}

\end{algorithmic}
\end{algorithm}

These steps ensure a wall is kept on the right hand side of the robot at
all times. The left-rule is simply the opposite where the robot turns left
if possible to keep the wall on the robot's left. Therefore,
this algorithm is inherently inefficient as it exhaustively searches the
maze. In many cases, one of these algorithms would be significantly quicker
than the other but which cannot be determined without prior knowledge of
the maze.

An alternative to the Wall Follower algorithm is Pledge's algorithm. This
aims to solve the problem where Wall Follower could be stuck in a loop. An
example of which is: if walls form to create a rectangle in the centre of
the maze, the robot would continually turn right or left (depending on
algorithm) around that rectangle, never finding the exit. Pledge solves this
by keeping a count of the number of turns made (if not right-angled turns,
then the angle of turn is used instead) as well as the initial direction of
travel~\cite{klein2011pledge}.

\begin{algorithm}
\caption{Pledge's Algorithm}
\begin{algorithmic}
\STATE Set angle counter to 0
\REPEAT
\REPEAT
\STATE Walk straight ahead;
\UNTIL {wall hit};
\STATE Turn right;
\REPEAT
\STATE Follow the obstacle's wall;
\UNTIL{angle counter = 0;}
\UNTIL {exit found;}
\end{algorithmic}
\end{algorithm}

Using the initial direction and counting the turns made, allows the
algorithm to avoid traps such as loops which simple wall followers can be caught in, such as an approximate ``G'' or ``6'' shape.

The Trémaux maze-solving algorithm requires the robot to record its
path and mark the routes it has taken throughout its navigation routine. Any
given path can be unmarked, marked once or marked twice. If a path is marked
twice that indicates the robot has travelled down it in both directions. The
robot will not travel down any path marked twice for a third time, therefore
treating them as dead-ends. Paths without markings are prioritised, before
choosing those marked once in search of further unmarked paths. If a
solution is found then the path that is marked only once is the route
from the goal back to the start point. Otherwise if no solution is found,
all paths in the maze will be marked twice \cite{even2011graph}.
Importantly, this will likely not find the shortest path, but it will be
guaranteed to find one if it exists.

\subsection{Path Finding}\label{litreview/maze/path}
There are many algorithms that can be used to find the shortest path
through a graph such as Dijkstra's algorithm and A* search. In the
context of this project, path finding is likely to be used once
the goal has been found to return the robot back to its starting
position. In the case of an unweighted graph (a graph where all edges are
equivalent to search), one of the simplest algorithms
is breadth-first search. This explores all neighbours at the same depth
before moving onto nodes at the next depth level. This would be guaranteed
to find a path with the shortest number of edges and is $\mathcal{O}(|V| + |E|)$
where $|V|$ is the number of vertices and $|E|$ is the number of edges~\cite{cormen2009introduction}.
Breadth-first search searches the oldest discovered node first meaning
many backtracks are needed to reach the oldest discovered node. As
backtracking in physical space is expensive, this results in a very
costly search for a single-agent. As more agents are added, the cost of this
decreases as less backtracking would take place.

In the case where the edges have a weighting, which in this case would be the
distance to next junction, Dijkstra's algorithm is one
of the most commonly used methods for finding the length of shortest path between a specified pair of nodes in the graph. The steps for Dijkstra's algorithm are outlined in Algorithm \ref{alg:Djikstra}.

\begin{algorithm}[!htb]
\caption{Dijkstra's Algorithm}
\label{alg:Djikstra}
\begin{algorithmic}
\REQUIRE{G = \{vs, es\} : G :: Graph, vs :: List(Vertex), es :: List(Edge)}
\REQUIRE{initial $\in$ vs : initial :: Vertex}
\REQUIRE{successors :: Vertex $\to$ List(Vertex); Finds vertices connected to input}
\REQUIRE{weight :: Edge $\to$ Int; Weighting factor applied to edge}
\REQUIRE{edge :: Vertex $\to$ Vertex $\to$ Edge; Finds edge which links vertices}
\FORALL{v $\in$ vs}
  \STATE{distance[v] $\gets \infty$}
  \STATE{visited[v] $\gets$ False}
\ENDFOR
\STATE {distance[initial] $\gets 0$}
\REPEAT
  \STATE{next $\gets$ v : distance[v] = min(distance) \AND visited[next] = False}
  \IF{no available next}
  	\RETURN distance
  \ENDIF
  \IF{distance[next] = $\infty$}
    \RETURN null
  \ENDIF

  \STATE{visited[next] $\gets$ True}
  \FORALL{v $\in$ successors(next) : visited[v] = False}
    \IF{distance[next] + weight(edge(next, v)) < distance[v]}
      \STATE{distance[v] = distance[next] + weight(edge(next, v))}
    \ENDIF
  \ENDFOR

\UNTIL{return condition reached}
\end{algorithmic}
\end{algorithm}

This version of Dijkstra's algorithm runs in $\mathcal{O}(|V|^{2})$
\cite{xu2007improved}. An improvement to this original algorithm using a
min-priority queue implemented using a heap, runs in $\mathcal{O}
(|E| + |V|log |V|)$ and was first proposed in~\cite{fredman1987fibonacci}.


Developed initially as an extension of Dijkstra's algorithm, A* is a
path-finding algorithm that typically achieves better performance by
use of a heuristic. At each node, A* chooses the node which it believes
can be used in the shortest path. It does so by choosing the node with
the lowest value of $f$, where $f$ is the sum of $g$ and $h$, $g$ being the
distance to the current node plus the known distance to that adjacent
node, and $h$ the heuristic value from that node to the goal. The heuristic
value can be found exactly as the first step of the algorithm though
this is very time consuming, so approximate heuristics are used such
as the Manhattan distance, Diagonal distance and the Euclidean distance.
