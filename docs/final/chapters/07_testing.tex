% !TEX root = ../report.tex

\chapter{System Testing}\label{systest}
System testing took place following integration testing and 
the full construction of the robot. The aim of system testing is to ensure that 
each of the component parts work as expected when working together, and to test 
and evaluate the system as a whole. A modular 
maze testing environment was used throughout integration and 
system testing to allow as many different maze configurations 
as possible to be tested. Throughout integration the modular maze was used as a 
``SLAM playground'', where various sized boxes or simple two area configurations 
were used to test the robot's capabilities in these environments and assess the 
SLAM maps built when using these configurations. 
\section{Modular Maze}\label{test/maze}
The modular maze environment was created with system testing of as many maze 
configurations as possible in mind. By creating a modular environment, the maze 
was also reusable in future iterations of the project, or similar projects which 
required a secure area to test autonomous vehicles. 
\subsection{Design}\label{test/maze/design}
\todo{some page formatting??}
With these broad specifications in mind, a list of requirements for the maze 
environment was created. 

\begin{table}[!ht]\centering
\caption{Modular Maze Requirements
\label{maze_reqs}}
    \begin{tabular}{ccc}
        \toprule
        \thead{Requirement} & \thead{Priority}\\
        \midrule
        Be easily modifiable & High\\
        Have enough cells to allow varied mazes & High\\
        Have minimum cell width $>$ diameter of the robot & High\\
        Be accompanied by sufficient ``pegs'' and ``walls'' to build complex 		mazes & High\\
        Wall material be non-translucent and non-reflective & Medium\\
        Base material be non-reflective & Medium\\
        Maze be easily transportable & Medium\\
        Maze base be modular & Low\\
        \bottomrule
    \end{tabular}
\end{table}

These requirements were then used to create a number of approximate ideas which 
were presented to the mechanical workshop. The maze base would be \SI{1280 x 
1690 x 18}{\mm}, giving 7 columns by 9 rows of peg holes each \SI{12}{\mm} in 
diameter and \SI{210}{\mm} from centre to centre. The holes drilled into the 
maze base would be \SI{15}{\mm} in depth leaving a \SI{3}{\mm} section of the 
base board un-drilled within the hole for the peg to rest on. The ``pegs'' would 
then be \SI{165}{\mm}, a protrusion of \SI{150}{\mm} from the top of the base 
board, with a cross slit cut from the top to roughly \SI{75}{\mm} down the peg 
(half the visible portion). These slits would be approximately \SI{3}{\mm} to 
allow the walls to be slotted in. In order for this to be accomplished, the 
walls would be winged --- \SI{208}{\mm} at the top and \SI{196}{\mm} \SI{75}
{\mm} from the top. This gives \SI{6}{\mm} spaces at either side to allow for 
the diameter of the peg and means the walls can be slotted between two pegs with 
ease. 

The materials required were researched in order to obtain the measurements 
above, such as \SI{12}{\mm} being a standard size for wood dowel and \SI{3}{\mm} 
and \SI{18}{\mm} respectively being depths of MDF which could be used for the 
walls and base boards. These requirements, descriptions and accompanying 
drawings were delivered to the mechanical workshop for construction to begin. 

\subsection{Implementation}\label{test/maze/impl}
The mechanical workshop required a number of changes to the original design in 
order for the maze to be created.

Foremost, the largest single item of solid wood which could be obtained from 
suppliers was smaller than the requested measurements of \SI{1280 x 1690 x 18}
{\mm}. Two resolutions were proposed to this by the mechanical workshop: reduce 
the width of each cell, or remove a column of holes from the maze. Following 
team consultation, it was decided that reducing the size of each cell would 
result in more flexibility in the end product, as even if the robot couldn't 
make a full turn in a single cell, 3 columns of two cell wide paths could still 
be created. It is also worth noting that the reduced cell width of \SI{195}{\mm} 
still met the requirement of the cell width being larger than the diameter of 
the robot. 

Secondly, the materials used for the walls and pegs would not be wood as this 
would involve a number of man hours to create the slits in the dowel and 
manually cut the walls to size. In order to streamline the process, it was 
suggested that the walls be made of perspex\todo{is this right??} --- to allow 
them to be laser cut --- and the pegs be 3D printed --- allowing mass printing 
once a design had been finalised. The group agreed that this was a good idea and 
in order for the pegs to be 3D printed they had to be made using CAD modelling 
software. None of the group had particular experience using this software, hence 
a steep learning curve was involved in carrying out this task. 

A peg matching the original specifications was modelled, however upon delivery 
to the mechanical workshop, the group were informed that this designed had 
changed to allow the walls to be cut as rectangles only --- with no wings. A new 
peg design was created which had slits at each \SI{90}{\degree} interval, 
running the length of the peg. This allowed the wall to be dropped in without 
the need for wings. This peg design was also modelled. 

Upon returning to the mechanical workshop the group were once again informed 
that the peg design had to be altered. The base board had been drilled through 
and so the pegs now required a smaller peg on the bottom and a rim to ensure 
that they did not fall through the base board. The agreed design was the same as 
previously with the new peg diameter as \SI{15}{\mm} and the small peg at the 
bottom, the original \SI{12}{\mm} diameter and \SI{15}{\mm} in length. The 
design was agreed and the peg was modelled. 

Midway through modelling the group were called by the mechanical workshop and 
informed that the perspex which had been ordered was slightly greater than the 
originally thought \SI{3}{\mm} and so the slits would have to be widened to 
\SI{4}{\mm} to allow for this. The model was altered to account for this change. 
This peg was delivered to the mechanical workshop and 3D printed. 

Unfortunately, once printed it was discovered that the design resulted in a peg 
which was unstable and often broke at the base. The mechanical workshop 
attempted to resolve this issue by increasing the density of the printing, 
however this was unsuccessful. Eventually, the mechanical workshop concluded 
that ordering a pre-slitted rod of aluminium and attaching metal pegs to the 
base to allow them to fit into the base board would be the best solution. This 
solution was successful and the maze with \todo{with XX pegs and XX walls} was 
delivered in full shortly afterwards. 

As can be seen from the table below the requirements laid out in the design 
phase (see Section~\ref{test/maze/design}) were largely met with the only two 
requirements not met being of medium or low priority.  

\begin{table}[!ht]\centering
\caption{Modular Maze Requirements
\label{maze_reqs_met}}
    \begin{tabular}{ccc}
        \toprule
        \thead{Requirement} & \thead{Priority} & \thead{Met}\\
        \midrule
        Be easily modifiable & High & Met\\
        Have enough cells to allow varied mazes & High & Met\\
        Have minimum cell width $>$ diameter of the robot & High & Met\\
        Be accompanied by sufficient ``pegs'' and ``walls'' to build complex 		mazes & High & Met\\
        Wall material be non-translucent and non-reflective & Medium & Met\\
        Base material be non-reflective & Medium & Met\\
        Maze be easily transportable & Medium & Not Met\\
        Maze base be modular & Low & Not Met\\
        \bottomrule
    \end{tabular}
\end{table}
\section{Testing Strategy}\label{systest/strategy}
The testing strategy designed for system testing was to develop a number of 
increasingly difficult mazes and test the capability of the robot to find a goal 
in each of these mazes. Additional robots will then be added to each of these 
configurations and the time recorded for quantitative testing of the project. 
Throughout the tests, qualitative results and conclusions will also be drawn to 
determine if the robots are co-operating effectively when searching the various 
maze configurations. 

A number of maze configurations were drawn up to be used as the default test 
cases for the robots. These configurations are shown below. \todo{add maze 
diagram} It is worth noting that these configurations were created before the 
cell width of the maze was reduced and therefore, maze configurations with 1 
cell width paths are significantly more difficult to traverse than previously 
thought. This will be taken into consideration when testing and adjusted 
accordingly if required. 


\section{Results}\label{systest/results}
Due to the issues described with SLAM and PID tuning described in Sections~
\ref{soft/SLAM/impl} and \todo{add PID tuning section}~\ref{} respectively, 
minimal system testing took place. The system tests which did take place were 
without SLAM/with poor SLAM and therefore could have no AI module as this is 
dependent on knowing the world state (having a map built). Despite this, the 
system as it was at the end of the project was tested on the maze configurations 
and the results are as follows. \todo{Are we gpoing to do this?}

\begin{table}[!ht]\centering
\caption{Modular Maze Requirements
\label{maze_reqs_met}}
    \begin{tabular}{ccc}
        \toprule
        \thead{Maze Configuration} & \thead{Number of Agents} & \thead{Time taken[\si{\second}]}\\
        \midrule
        1 & 1 & xx\\
        1 & 2 & xx\\
        2 & 1 & xx\\
        2 & 2 & xx\\
        3 & 1 & xx\\
        3 & 2 & xx\\
        4 & 1 & xx\\
        4 & 2 & xx\\
        \bottomrule
    \end{tabular}
\end{table}

The results shown are not indicative of the project had it been completed in its 
entirety, however these results give benchmark scores for the shown maze 
configurations with what is essentially a random AI module on top. In practice 
the robots performance is slightly worse than this as there is also no mapping, 
and so a random AI could perform better. This is because in this case choices 
regarding paths would still be being made and paths would not be being repeated 
which can be the case in these results. Work detailed in further work (Section~
\ref{furtherwork}) can use the results shown here as benchmark results and 
measure the success of SLAM and AI implementations against these results to 
evaluate progress being made.  \todo{add more when we have the results}
