% !TEX root = ../report.tex

\begin{abstract}
\noindent
This project aims to develop co-operative robotic systems which effectively map and search an area using non-telemetric communication. This will allow the systems to perform even in environments which hinder telemetry, such as caves or tunnels. Each of these systems should be able to solve this problem individually; however, after the introduction of multiple robots to the area, they should be able coordinate their movements to distribute the task and the problem should be solved faster.
The robots will be tested in a custom-made modular testing environment, which will be used to construct simple mazes in which a target is to be located, and evaluated based on
their performance both individually and co-operatively.
Detailed analysis of the electrical, software and mechanical design solutions made
is presented. The robots will use a differential drive system and will be designed
and constructed with the necessary hardware to complete the task of exploring
their maze-like environment in a distributed fashion. Computer vision systems
on the robots will be used in conjunction with ultrasonic sensors to perceive the
environment and identify other robots in the group, and incremental encoders and an
inertial measurement unit will be used to track the robots position and orientation.
Communication will be conducted over Wi-Fi systems, which
will be artificially restricted to require line-of-sight, so as to simulate situations
where global communication is not possible.
The report also contains extensive discussion of the project management
approach used to properly coordinate the project team. The structure of the
team and practices employed have been considered carefully and a detailed timeline
developed to ensure minimum project risk.
\todo{Testing outcomes}
The outcomes of the testing were
\todo{Evaluation}
The project was evaluated as being overall success/failure.
\end{abstract}
