% !TEX root = ../report.tex

\begin{abstract}
\noindent 
%This is 283 words as of 19:08 Thursday, don't think we should add much to it length wise

Several co-operating agents can coordinate to solve many tasks faster than a single agent.
However, most existing multi-agent systems involve centralised control and telemetric
communication. This project aimed to develop co-operative robots which effectively map and 
search an area using non-telemetric communication and distributed control. This will allow 
therobotic systems to perform even in environments which hinder telemetry, such as
caves or tunnels. Each agent was to be able to solve this problem 
individually; however, after the introduction of multiple robots to the area,
they should be able to distribute the task to solve it quicker. The robots will
be tested in a custom-made modular testing maze environment, in which a target
is to be located, and their performance evaluated both individually and
co-operatively. The robots use incremental encoders and an inertial measurement
unit are used to track the robots position and orientation as it moves through its environment.
Computer vision is used in conjunction with ultrasonic sensors to perceive the environment,
identify other robots and produce a map of the area. Communication is conducted over 
ad-hoc Wi-Fi, which is artificially restricted to require  line-of-sight between robots, so
as to simulate situations where global communication is not possible. The report presents a
detailed analysis of these electrical, software and mechanical design solutions as well as
an extensive discussion of the project management approach and practices employed used to
properly coordinate the project team and minimise potential risk. The rigorous testing process
of each subsystem and higher level system testing is also described. Unforeseen setbacks and
technical difficulties delayed the completion of some latter objectives which has resulted in
the AI module not being fully integrated. The system  was, however, implemented to a high standard,
and its modularity would allow it could easily be continued in the future.
\end{abstract}
