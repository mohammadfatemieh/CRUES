% !TEX root = ../report.tex

\chapter{Evaulation}\label{eval}
Based on the original objectives of the project (see Section~\ref{intro/
objectives}) the project can be deemed as somewhat of a success. 4 out of 6 of 
the original ``Major'' objectives have been completed in addition to 1 of 2 of 
the ``Optional'' objectives. In addition to this, it is believed that with a 
small number of additional weeks, the two remaining ``Major'' objectives could 
be completed. Given the project as a whole aimed to combine 3 engineering 
disciplines and tackle a number of complex problems simultaneously, a strong 
effort \todo{This doesn't sound right, couldn't think what did} has been made. 
As detailed in various previous sections of the report, a number of small setbacks in 
various aspects of the project caused the latter stages of the project to be 
uncompleted. This was magnified as many of the latter elements were sequential, 
thus one failure (SLAM) was compounded into two. 

Despite these failures, the management of the project was effective and the 
asynchronous tasks were completed largely to schedule in the early to middle 
stages of the project. Less management of the project was required in the latter 
stages as objective related tasks became sequential, and those who weren't 
working on those tasks were assigned secondary objectives such as those 
pertaining to the trade show or the report. Git was used effectively, with the 
issues feature and protected merging used to ensure all members of the team were 
working on a task and completion of tasks could be monitored by the project 
manager. 

A series of benchmark results have also been obtained by can be used by future iterations of the project to measure success. The 
objectives which have been completed were also iterated upon multiple times to complete for each of the robots and hence the design has 
been improved by using an iterative design process. 


\todo{Overall eval in here before individual sections}
\todo{evaluate pm structure and agile methods etc.} 
\section{Mechanical}\label{eval/mech}
Despite a lack of prior knowledge in the mechanical engineering field, the 
mechanical aspects of the project have been completed to a 
high standard. This is mainly due to the decision to complete these tasks first 
with a higher priority and allow time for the designs to 
be iterated upon throughout the project. The first robot was measured for each 
of the parts required to mechanically fit the robot 
together and this was improved between robots, leading to a final construction 
which was consistent and simple. Given the eventual 
incompletion of the project, the decision taken to purchase a pre-built chassis 
as opposed to create a bespoke chassis was justified. 
This would have added unnecessary complication and a blocking task to the 
beginning of the project which could have caused future tasks 
to be delayed further. 

Aside from the robots, the other main mechanical component was the modular maze testing environment. Although constructed by the mechanical workshop, a great deal of effort was put into the design and subsequent iterations to obtain the best outcome possible. The modular maze is a reusable ``SLAM playground'' and is a major successful outcome of the project. Although not particularly transportable, the maze is robust and durable and intended to last --- and be used --- for future projects.

In addition, although not used in the final design a great deal was learned from using CAD to create designs of pegs which were intended for use in the maze. It was also worth noting that the CAD designed pegs were unusable, not due to a design flaw, but a flaw in the printing/construction process.  


\todo{talk about success of maze and CAD}  

\section{Electrical}\label{eval/elec}
Each of the sensors and actuators of the robot have been researched fully and a 
parts list created for ease of replication. The parts chosen meet the 
specification and carry out the task required to the required specification. 
Each of the modules of the robot --- sensors and actuators --- has a fully 
tested and integrable ROS node associated with it. The sensors and actuators are 
also fully connected as part of the mechanical design of the robot with the PCB 
designed for this purpose. The PCB design was iterated upon and therefore 
improved over the course of the project. Having multiple robots allowed an 
iterative design process to be carried out and this greatly improved the outcome 
of the PCB. Connections and vias were altered following the first design to be 
more easily integrable with the mechanical design and mean the final robot is 
constructed of a tried and tested final design. 

Time and care was take to ensure the electrical design of the PCB and any other 
connections between parts on the robot --- such as the ribbon cable connection 
between the PCB and the Raspberry Pi --- were robust and correct. This resulted 
in a robot which is both electrically and mechanically sound and replicable 
easily with time and parts. The robots which possess prototype parts, used in 
interim stages of the iterative design process, remain functional and usable. 
This demonstrates the careful selection and design of each of the parts. 

\todo{talk about all the good components and their code. Talk about successful PCB} 

\section{Software}\label{eval/soft}
\todo{write software eval}
