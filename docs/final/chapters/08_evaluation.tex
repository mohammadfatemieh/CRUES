% !TEX root = ../report.tex

\chapter{Evaulation}\label{eval}
Based on the original objectives of the project (see Section~\ref{intro/
objectives}) the project can be deemed as somewhat of a success. 4 out of 6 of 
the original ``Major'' objectives have been completed in addition to 1 of 2 of 
the ``Optional'' objectives. In addition to this, it is believed that with a 
small number of additional weeks, the two remaining ``Major'' objectives could 
be completed. Given the project as a whole aimed to combine 3 engineering 
disciplines and tackle a number of complex problems simultaneously, a strong 
effort \todo{This doesn't sound right, couldn't think what did} has been made. 
As detailed in various previous sections of the report, a number of small setbacks in 
various aspects of the project caused the latter stages of the project to be 
uncompleted. This was magnified as many of the latter elements were sequential, 
thus one failure (SLAM) was compounded into two. 

Despite these failures, the management of the project was effective and the 
asynchronous tasks were completed largely to schedule in the early to middle 
stages of the project. Less management of the project was required in the latter 
stages as objective related tasks became sequential, and those who weren't 
working on those tasks were assigned secondary objectives such as those 
pertaining to the trade show or the report. Git was used effectively, with the 
issues feature and protected merging used to ensure all members of the team were 
working on a task and completion of tasks could be monitored by the project 
manager. 

A series of benchmark results have also been obtained by can be used by future iterations of the project to measure success. The objectives which have been completed were also iterated upon multiple times to complete for each of the robots and hence the design has been improved by using an iterative design process. 


\todo{Overall eval in here before individual sections}
\todo{evaluate pm structure and agile methods etc.} 
\section{Mechanical}\label{eval/mech}
Despite a lack of prior knowledge in the mechanical engineering field, the mechanical aspects of the project have been completed to a high standard. This is mainly due to the decision to complete these tasks first with a higher priority and allow time for the designs to be iterated upon throughout the project. The first robot was measured for each of the parts required to mechanically fit the robot together and this was improved between robots, leading to a final construction which was consistent and simple. Given the eventual incompletion of the project, the decision taken to purchase a pre-built chassis as opposed to create a bespoke chassis was justified. This would have added unnecessary complication and a blocking task to the beginning of the project which could have caused future tasks to be delayed further. 


\todo{talk about success of maze and CAD}  

\section{Electrical}\label{eval/elec}
\todo{talk about all the good components and their code. Talk about successful PCB} 

\section{Software}\label{eval/soft}
\todo{}
