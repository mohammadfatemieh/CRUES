% !TEX root = ../report.tex
\chapter{Project Management}\label{pm}

The project was managed throughout by an elected project manager 
(Andrew Fagan), who was tasked with ensuring tasks were 
completed to the schedule decided by the group as a whole. This 
allowed every member in the group to have an equal voice in 
decisions regarding the project while still having a overall 
manager of the timeline. Using this system, with a minimal 
number of managers and no strictly assigned teams provided
workflow fluidity and flexibility and thus no work was restricted 
to a specific subset of the group. This also improved the 
group's understanding of the project as a whole and the aspects 
which will be detailed herein. 

\section{Project Structure}\label{pm/structure}

The project was structured in two week agile sprints. At the 
beginning of each sprint a number of tasks were created which 
were expected to be completed within the next sprint. These were 
then assigned to group members as appropriate. Upon completion 
of a task within the sprint, this was reported to the project 
manager and a further task was assigned. The project manager was 
responsible for ensuring estimated timelines were adhered to 
within the sprint. At the end of the sprint a short meeting was 
held to discuss progress made and issues encountered which would 
be used to influence the tasks for the subsequent sprint. Tasks 
within the sprint were tracked using Git issues, assigned to one 
or many members of the group. The agile methods used were 
generally well adhered to and 
resulted in successful outcomes for the tasks which they were 
used for. 
	

\subsection{Consultations}\label{pm/consultations}
Throughout the project a number of members of academic staff 
were consulted to obtain an expert's opinion on either 
discussions or decisions which had been made by the group. 
Members of staff from the EEE, CIS and EMEA departments were 
consulted at various stages in the project, bringing together 
each of the individual components of the robot. 

From the EMEA department, Dr Mark Post (now at York University) 
was consulted in the planning stages of the project. With a 
background in robotic vehicles and SLAM, Dr Post was consulted 
with a view to obtaining as much knowledge as possible 
surrounding the possibilities for operating system to use, 
possible SLAM implementations, potential objectives and 
timescales for parts of the project. Only one meeting took place 
as Dr Post moved to York University shortly after the meeting, 
however, the meeting was highly beneficial and gave a vast 
quantity of insight into existing solutions and software and 
also potential pitfalls which could be encountered throughout 
the project. Full notes available in Appendix A. \todo{Is this 
in depth enough? Do we want more info here? Also add notes to 
appendix}

Dr James Irvine, EEE Department, was consulted informally  
regarding the use of Sockets for communication using a Wireless 
Ad-hoc Network (WANET). Dr Irvine was approached with an 
assumption that using the Python Socket library to send 
serialised data between the Raspberry Pis would be a sufficient 
method of communication between the robots. This was confirmed 
by Dr Irvine with the additional suggestion that JSON could be 
used to serialise the data. 


Dr Gordon Dobie, EEE Department, was consulted towards the end 
of the project regarding the tuning of the PID controller. A 
meeting was arranged and the processes followed explained at the 
beginning of the meeting. Dr Dobie went on to say that although 
PID was useful, a more simple PD rate control system 
\ref{litreview/control/ratecontrol} would be more appropriate 
for the system. Following the in-depth investigations made into 
PID controllers, the group was fairly confident following some 
basic guidance, provided by Dr Dobie, on implementing the 
recommended system. 


Prior to the arrangement of the meeting with Dr Dobie, Dr Phil 
Rogers was consulted regarding the tuning of the PID controller. 
He suggested providing a frame of reference in order to correct 
the PID controller and maintain a straight line with regards to its reference frame. While the group waited on the meeting with Dr Dobie, who is a specialist in PID control, Dr Rogers advice was followed and SLAM was progressed in the interim.   

\subsection{Mechanical Workshop}\label{pm/mechshop}
The construction of the modular maze used for testing \ref{mech/
maze} was outsourced to the mechanical workshop located in the 
EEE department at the university. A number of discussions were 
had with the mechanical workshop to specify the work correctly 
before construction began however despite this, issues were 
encountered throughout the construction of the maze. 

Following an initial meeting where the idea was discussed and 
rough drawings were requested in order to clarify the 
specification, sketches of the 'pegs', 'boards' and 'base' were 
created and taken to the mechanical workshop. These sketches 
were deemed insufficient and more precise technical drawings 
were requested. These were provided and an understanding was 
achieved. 

Initially it had been thought that the 'pegs' would be made from 
existing wood dowel however, the mechanical workshop advised 
that 3D printing may be more efficient and CAD models were 
created of 3 different iterations of the 'pegs'. Each of these 
was presented in turn to the mechanical workshop and printed 
before it was discovered that revision was required. Upon the 
final iteration, the mechanical workshop advised that a 
different approach could be taken and that aluminium rods with 
existing slots could instead be used when cut into sections. 
This simplified the manufacturing process and the maze was 
delivered by the mechanical workshop shortly afterwards. The 
completed maze with an arbitrary configuration is shown below.
\todo{add figure of maze}
\todo{this might be better in the maze design and implementation section in more detail}  



\section{Timeline}\label{pm/timeline}
\todo{put a gantt chart here}



\section{Risk Evaluation}\label{pm/riskeval}

