% !TEX root = ../report.tex
\chapter{Project Management}\label{pm}

The project was managed throughout by an elected project manager (Andrew Fagan), who was tasked with ensuring tasks were completed to the schedule decided by the group as a whole. This allowed every member in the group to have an equal voice in decisions regarding the project while still having a overall manager of the timeline. Using this system, with a minimal number of managers and no strictly assigned teams allowed workflow fluidity and flexibility meaning no work was restricted specifically to a subset of the group. This also improved the group's understanding of the project as a whole and the aspects which will be detailed herein. 

\section{Project Structure}\label{pm/structure}

The project was structured in two week agile sprints. At the beginning of each sprint a number of tasks were created which were expected to be completed within the next sprint. These were then assigned to group members as appropriate. Upon completion of a task within the sprint, this was reported to the project manager and a further task was assigned. The project manager was responsible for ensuring estimated timelines were adhered to within the sprint. At the end of the sprint a short meeting was held to discuss progress made and issues encountered which would be used to influence the tasks for the subsequent sprint. 
	
	\todo{evaluate this method (that we definitely used throughout...all the time...stuck to it like glue)} 

\subsection{Consultations}\label{pm/consultations}
Throughout the project a number of member of academic staff were consulted to obtain an expert's opinion on either discussions or decisions which had been made by the group. 
\todo{Mark Post discussion}
\todo{James Irvine discussions}

\subsection{Mechanical Workshop}\label{pm/mechshop}
The construction of the modular maze used for testing \ref{mech/maze} was outsourced to the mechanical workshop located in the EEE department at the university. A number of discussions were had with the mechanical workshop to specify the work correctly before construction began however despite this, issues were encountered throughout the construction of the maze. 

Following an initial meeting where the idea was discussed and rough drawings were requested in order to clarify the specification, sketches of the 'pegs', 'boards' and 'base' were created and taken to the mechanical workshop. These sketches were deemed insufficient and more precise technical drawings were requested. These were provided and an understanding was achieved. 
	Initially it had been thought that the 'pegs' would be made from existing wood dowel however, the mechanical workshop advised that 3D printing may be more efficient and CAD models were created of 3 different iterations of the 'pegs'. Each of these was presented in turn to the mechanical workshop and printed before it was discovered that revision was required. Upon the final iteration, the mechanical workshop advised that a different approach could be taken and that aluminium rods with existing slots could instead be used when cut into sections. This simplified the manufacturing process and the maze was delivered by the mechanical workshop shortly afterwards. The completed maze with an arbitrary configuration is shown below.
\todo{add figure of maze}
\todo{this might be better in the maze design and implementation section in more detail}  



\section{Timeline}\label{pm/timeline}
\todo{put a gantt chart here}



\section{Risk Evaluation}\label{pm/riskeval}
