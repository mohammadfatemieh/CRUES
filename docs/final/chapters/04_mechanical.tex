% !TEX root = ../report.tex

\chapter{Mechanical}\label{mechanical}

The mechanical design of each robot is central to its functionality as several of the sensors are
reliant on the accuracy of the mechanical construction. As multiple robots are being created, 
the mechanical similarity is also important to guarantee the sensors and software function consistently across each robot. 
Given this consideration, Printed Circuit Boards (PCBs) were in the
final iteration of the design---with strip board being used for prototyping---as this will provide additional robustness and guaranteed repeatability between robots.

\section{Chassis}\label{mech/chassis}
The chassis used for the robots was the pre-built ``hobbyist'' chassis from Pololu ~\cite{pololuchassis}. These chassis were chosen as they ensured manufacturing  consistency and allowed for scalability, due to their availability, both of which were essential design consideration. The chassis were accompanied by a number of fitted parts such as motors, encoders and a power distribution board which were also used. Using these pre-built solutions reduced the number of parts which had to be created and fitted, simplifying the construction process.
The chassis had numerous mounting holes, which again eased construction, as holes did not need to be drilled in the chassis to mount the PCB and other components. A variety of chassis colours were also used to simplify the robot recognition which was required and will be discussed later in Section ~\ref{software/cv}. 

During testing, issues were encountered with some of the motors which had accompanied the chassis, as it was noticed that they were leaking oil or grease from their plastic casing. This resulted in the gears making more noise than they did previously and resulted in motor failure if the motors were run without more grease for a prolonged period of time. This was investigated through the manufacturer, however, unfortunately no reply has been forthcoming. The risk of motor failure was reduced by applying a more appropriate grease at the recommendation of \todo{insert Steve's surname????}, the lab technician.  

\section{Sensor Placement}\label{mech/sensors}
The range sensors chosen were ultrasound sensors as these fit the scalability and accuracy requirements of the range sensing. Mechanically these had to be mounted in a consistent manner on each robot to ensure that software results could be replicated between robots with minimal deviation from the standard code. The ultrasonic sensors chosen were HC-SR04 \todo{cite us sensor datasheet}.

As can be seen from the datasheet, the sensors have a cone of detection of \ang{30}. This was the main consideration when designing the layout of the sensors at the front of the robot. Due to space restrictions on the front of robot, three sensors were used for an overall cone of detection of approximately \ang{90}. This was first drawn roughly on a scale drawing of the robot before the cones of detection were tested using stripboard\ref{elec/range}. Following these tests, the PCB could be designed with these measurements in mind\ref{elec/PCB} and headers used to ensure the design was modular and sensors could be swapped out if needed. 

The inertial measurement unit (IMU) had to be placed in the centre/as close to the centre of the robot as possible. This was to ensure the the IMU was as close to the centre rotation of the robot as possible, to obtain the most accurate readings possible. By placing the IMU correctly on the robot the need for regular calibration of the measurements is reduced. This heavily influenced the design of the PCB. 


\section{Drive System}\label{mech/drive}

The drive system for the robots will be a differential drive system (DDR).
This drives each wheel independently using independent actuators and the
wheels are not connected by a single axle~\cite[p.~146]{braunl_embedded_2013}.
When using a two wheeled robot, DDR allows the
robot to rotate on the spot around its central axis when the wheels are
driven in opposite directions. This provides a high level of mobility which
will aid in the sensing and mapping capabilities of the robot. The chassis chosen is accompanied by a motor and encoder for each wheel which can be used to obtain data for wheel odometry .
