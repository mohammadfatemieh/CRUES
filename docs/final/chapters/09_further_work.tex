% !TEX root = ../report.tex

\chapter{Further Work}\label{furtherwork}
As not all of the major objectives of the project were completed, the main 
aspect of further work is to implement various AI algorithms and investigate 
their effectiveness in solving mazes co-operatively.  A series of benchmark 
results have been provided in Section~\ref{systest/results} to evaluate 
these by. It is worth noting that A* is likely to be the best path planning 
algorithm once the maze has been fully or mostly explored. In order to 
explore, a number of different algorithms could be used. Starting with a 
frontier-based algorithm, using periodic scans to improve the SLAM map, this 
could be built upon using a depth first approach to avoid doubling back. 
Robot behaviour on goal discovery would also have to be , explicitly: whether 
one goal being discovered was an end state --- meaning the robot could return 
to its starting point --- or whether more potential goals should be searched 
for.
  
Various improvements could also be made to the SLAM (see Section~
\ref{soft/SLAM}) and Odometry (see Section~\ref{soft/odometry}).  The 
exploration through testing of other libraries and methods of improving this 
system when using ultrasonic sensors and limited feedback methods, could take 
place and may provide improvements to the mapping capabilities of the robot. 
Two such libraries are: SLAMOT (SLAM with Object Tracking) which could be 
used to resolve the issue of robots existing in each other's maps as moving 
objects would be tracked and remain unmapped; and AMCL for localisation, 
which compensates for the drift in time between the odometry and reference 
frames.   

The second optional objective, functionality for map sharing and SLAM loop 
closure 
across robots, could also be implemented. This would allow robots which have covered 
different paths to communicate their knowledge of the maps and improve search 
efficiency. It would also allow both robots to improve the accuracy of their existing 
map by combining it with that of the other robot. 

Subsequent to these, more robots could be added to the system in order to 
stress test the scalability of the 
project and its components such as ultrasonic synchronisation, 
communications and path-finding. If these stress tests were successful, 
the maze environment could be altered to have walls and a floor which more 
closely simulate a real life environment. These changes may include adding 
texture to the walls to evaluate the response of the ultrasonic sensors; 
using the maze in a darkened environment to evaluate the response of the 
computer vision node and if additional lighting is required on the robots; 
and altering the texture of the floor to evaluate the response of the motor 
control system and the stability of the robots. 

Following these evaluations, the system could then either be upscaled in 
mechanical terms to perform in the field or used directly and tested in the 
field. Explorations of different real-life environments could be evaluated 
and the challenges which come with those tackled. This could lead the 
project to being a cheap, modular option which could be used in a variety of 
applications such as those described in this report. 

Using a similar software and sensing system, underwater or aerial agents 
could also be created and tested, further evaluating the software system's 
effectiveness and the overall modularity of the system. 