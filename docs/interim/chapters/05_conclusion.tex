% !TEX root = ../report.tex

\section{Conclusion}\label{conclusion}
\thispagestyle{plain}

The aim of this project is to develop a co-operative robotic system capable of
searching an environment in a coordinated and distributed manner, using exclusively
non-telemetric communication.

The robots will use a differential drive system mounted on a pre-built chassis.
This will both minimise the time required to build the robot and provide more
consistent frames than could be achieved using bespoke solutions at a similar
budget. This also makes it far more straightforward to mount the motors and encoders
as they are designed specifically for the chassis chosen.

Several ultrasonic range finders will be positioned along the front of the robot
to take distance measurements to the surrounding walls and obstacles in the
environment. Over time these measurements will be used to develop a map of the
surrounding area. This will be augmented using computer vision system capable
of object identification, which will identify moving objects (that should not
be mapped) such as other robots, and allow loop closure though recognising
distinctive areas in the environment. The computer vision will also reduce the
communication required, as robots will be able to observe other robots without
direct communication. The object identification will likely be feature-based,
probably either ORB or BRISK due to their computational efficiency relative to
other feature-based algorithms. The The robots' positions and orientations will
be tracked using wheel odometry and 6 DOF IMU data which will be combined using
a sensor fusion algorithm.

The team has appointed a project manager responsible for ensuring project
deliverables, decided upon several workflow practices, and developed a detailed
project timeline to ensure there is minimal project risk.
